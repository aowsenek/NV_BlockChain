\chapter{Anchor Blocks}

So far we have covered blockchain, how to implement it, and the benefits it brings to logging applications. However, computer logs, and vehicular logs in particular, need to be able to be deleted without compromising the entire blockchain. Logs need to be periodically deleted as they take up storage space and after a certain point are no longer useful to keep. A blockchain implementation for logging verification needs to be flexible enough to scale with, say, vehicle fleet sizes and the immense amount of data onboard computers generate. Thus we introduce the anchor block. An anchor block is a block in the blockchain placed every set number of block (e.g. Every 50 to 100 blocks.). Their purpose is to provide a point in the blockchain where we can remove all the preceding data and blocks, and keep teh blockchain still valid. 
Previous iterations of the anchor block were designed around having the anchor block contain no sensitive data, however this is an issue because every anchor block, the chain functionally restarted. There was nothing truly tying the contents of the chain before the block to the contents of the chain after the block, and was thus more vulnerable to 51 percent attacks and other blockchain attacks that are thwarted because of the long nature of blockchain.
In the current iteration of anchor blocks, anchor blocks are also standard data blocks. This means they contain the log data any other block holds, but also an additional parameter that specifies that the block is an anchor block. This means there is meaningful linkage to the chain before the anchor block and the chain after the anchor block, provides faster vaildation checking, and allows data after the anchor block to be deleted if necessary.
With this implementation, it is possible to periodically delete the blockchain and corresponding logs up to an anchor block. Because the anchor block is the first one in the chain, it is assumed to be a vaild block unless previously invailidated. An anchor block can be invaildated when the data it contains does not match up the the logs that exist. This could happen in any of the three attacks, but would be detected before the invalid anchor block becomes the first in the chain. It is important to mention that the anchor blocks allow mass deletion of data before them, not selective data deletion that an attacker could use to omit important information from the ally. 
Of course, the nature of anchor blocks means that the more blocks between anchor blocks, the more effort it takes an attacker to attempt to overcome the chain and replace it with their own. But the increased security of longer sections also comes with an increased tradeoff- more blocks means more storage necessary. This could be mitigated by not holding on the data very long, however that also means trading off a decrease in "insurance." Ultimately this tradeoff is for any organization that implements the blockchain logging software to decide what makes sense for their use case. An organization with a fleet of vehicles may only want to keep the data for a month and save any data they actively need (Say, a court case over a car crash) elsewhere. An organization that handles sensitive government data may want to keep any logs for several years in case it is discovered that they have been compromised. 

This blockchain use case could also help when attempting to discover a root kit, as the root kit will attempt to hide discovery of itself in logs by deliberately not writing its existence out. However, when a use takes the logs and attempts to verify they are correct, the blockchain and logs will not verify and the user will know log tampering has taken place.

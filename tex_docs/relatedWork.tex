\chapter{Related Work}
\label{relwork}
\subsection{Blockchain Overview}
        %What
        Blockchain is type of public ledger that records all transactions and authenticates previous transactions via its peer-to-peer attributes. Transactions entered into the blockchain cannot be erased or removed. Unfortunately, blockchain is often mistaken as analogous with digital currencies such as Bitcoin, Etherium, etc. While these currencies do depend on blockchain as a fundamental reason for their ability to be decentralized, blockchain is a technology used in the currencies, but the currencies are not blockchain. 
        
        %Timeline
        The concept of blockchain was initially developed and introduced to the world through the Bitcoin white paper by Satoshi Nakamoto, published October 31st, 2008 \cite{Bitcoin}. The first transaction recorded on the blockchain was a transfer of 10 Bitcoins from Nakamoto to a programmer and first adopter, Hal Finney, on January 9th, 2009. As Bitcoin gained popularity, other developers adopted the blockchain ledger system for distributed currency transactions. In 2016, an article in the Applied Innovation journal was published describing the many non-financial uses of blockchain ledgers \cite{BlockchainTechUses}. As people have begun to realize the uses of this technology, more ideas are being created. While there are projects using blockchain for non-cryptocurrency uses that are currently deployed, none of them have reached widespread public use. In fact, as Yli-Huumo et. al. reported, of 41 scientific papers on blockchain, only 20\% of them focus on blockchain applications unrelated to Bitcoin \cite{BlockchainResearch}.
        
        %Where it is used
        There have been many implementations of blockchain, starting initially with Bitcoin \cite{Bitcoin}.
        The early blockchain algorithm, Bitcoin, sought to solve the primary issue of a decentralized currency- how can users trust the currency without relying on a central regulatory body?  This approach led to creating a chain of "blocks" or a blockchain, that both log data and record and authenticate all previous blocks in the chain. In Bitcoin, this approach is used to verify, authenticate, and log currency transactions. Because of its distributed nature there are such issues as multiple branches of a chain, and because of its nature as a currency, security issues such as double-spend attacks \cite{Bitcoin} that the Bitcoin white paper attempts to address and solve. I do not anticipate either multiple-branch issues or double-spend attacks as an issue with a single-node blockchain that uses logs instead of currency transactions. 
        
        A second paper, Decentralizing Privacy: Using Blockchain to Protect Personal Data by Zyskind et al. is about a decentralized blockchain for protecting personal data \cite{PrivacyChain}. Zyskind et al. discuss a bitcoin-based logging software for privacy of user data, which is intended to provide a trusted decentralized log of whose data was accessed when, which is stored in the form of data-access instructions. \cite{PrivacyChain}. 
        
         A third algorithm, Logchain, also tried to develop an implementation of a blockchain-based log verification software. However, despite the technology existing and being understood, it has not been implemented in "the wild." This is because it is very expensive to run, as all the data from the log is put onto the Etherium chain, which rapidly inflates the costs of running the software. \cite{Logchain} We seek to extend their work by reducing the amount of data needed to put on the blockchain to ensure log validity, thus making the software cheaper and more practical for real-world use. 
         
        \subsection{Vehicle Log Data}
        Telematics is a term that refers to devices that combine wireless data communication and infomatics. In vehicles it is used to find GPS location, and record and transmit vehicular state data such as gas level, miles per gallon, vehicular weight, etc. Aftermarket telematics devices (ELDs) that can be plugged into vehicles are very popular in large-scale vehicle fleets as the easiest way to report the telemetry data to the company. 

        In December 2015 a regulation was created by the USDOT that said all qualifying vehicles must be equipped with an ELD or AOBRD (automatic onboard recording device) by December 2016. By December 2019, all qualifying vehicles must be equipped with an ELD \cite{ELDRULE}. Before these regulations were approved, all vehicles that were required to log activity did so on paper.
        
        \subsection{Neutral Vehicle}
        According to the Neutral Vehicle website, "Neutral Vehicle is a group that aims to create an international standard for vehicle logs. The Neutral Vehicle platform provides an end-to-end framework for transferring rich vehicle data from the ground to the cloud and back. The development of advanced applications and services by third parties becomes possible through a platform constructed with security, scale and interoperability at its core\cite{NeutralVehicle}." Working with the Neutral Vehicle team will grant us access to logging data ad hardware that would otherwise be inaccessible.

